\section{Inverse Kinematics} \label{sec:inverse_kinematics}
Before solving the inverse kinematics problem, we need to find the singular confugurations of the manipulator
using the Jacobian matrix. The singular configurations are the configurations where the determinant of the Jacobian matrix is $0$

\subsection{Singular configurations} \label{subsec:singularity}
\subsubsection*{Geometric Jacobian}
All data necessary to compute the geometric Jacobian can be
extracted from the matrices derived while solving the Direct Kinematics
Problem.
\begin{equation} \label{eq:J}
    \mathcal{J}=\begin{bmatrix}
        \mathcal{J}_v \\
        \mathcal{J_\omega}
    \end{bmatrix}
\end{equation}

\begin{equation} \label{eq:Ti0}
    {_i^{0}}T=\begin{bmatrix}
        {_i^{0}}r_{11} & {_i^{0}}r_{12} & {_i^{0}}r_{13} & {_i^{0}}r_{i3} & {_i^{0}}P_x \\
        {_i^{0}}r_{21} & {_i^{0}}r_{22} & {_i^{0}}r_{23} & {_i^{0}}r_{i3} & {_i^{0}}P_y \\
        {_i^{0}}r_{31} & {_i^{0}}r_{32} & {_i^{0}}r_{33} & {_i^{0}}r_{i3} & {_i^{0}}P_z \\
        0              & 0              & 0              & 1              & 1
    \end{bmatrix}
\end{equation}
\begin{equation}
    \implies{_i^{0}}P=\begin{bmatrix}
        {_i^{0}}p_x \\
        {_i^{0}}p_y \\
        {_i^{0}}p_z
    \end{bmatrix}
    and \quad
    {_i^{0}}z=\begin{bmatrix}
        {_i^{0}}r_{13} \\
        {_i^{0}}r_{23} \\
        {_i^{0}}r_{33}
    \end{bmatrix}
\end{equation}
All joints of the manipulator are revolute. Threfore,
\begin{equation}
    \begin{split}
        \mathcal{J_\omega}&=\left[{_1^{0}}z \quad {_2^{0}} \quad z{_3^{0}} \quad z{_4^{0}} \quad z{_5^{0}} \quad z{_6^{0}}z\right]\\
        \mathcal{J}_v&=\left[{_1^{0}}z \times \left({_6^{0}}p-{_1^{0}}p\right) \quad {_2^{0}}z \times \left({_6^{0}}p-{_2^{0}}p\right) \quad
        {_3^{0}}z \times \left({_6^{0}}p-{_3^{0}}p\right) \quad {_4^{0}}z \times \left({_6^{0}}p-{_4^{0}}p\right) \quad
        {_5^{0}}z \times \left({_6^{0}}p-{_5^{0}}p\right) \quad {_6^{0}}z \times \left({_6^{0}}p-{_6^{0}}p\right)\right]
    \end{split}
\end{equation}
The determinant of the Jacobian matrix is computed using MATLAB's symbolic toolbox.
\begin{equation}
    \det\left(\mathcal{J}\right)=-a_2s_5\left(a_3s_3 + d_4c_3\right)\left(a_3c_{23}-d_4s_{23} + a_2c_2\right)
\end{equation}
Setting $\det\left(\mathcal{J}\right)=0$ results the following.
\begin{equation} \label{eq:singular_config}
    \begin{aligned}
         & a_2 = 0                            \\
         & s_5 = 0                            \\
         & a_3s_3 + d_4c_3 = 0                \\
         & a_3c_{23} - d_4s_{23} + a_2c_2 = 0
    \end{aligned}
\end{equation}
Now that we have established the singular configurations, we can proceed to solve the inverse kinematics problem.
Let the desired position and orientation of the end-effector be given by ${_6^{0}}T_d$ as:
\begin{equation} \label{eq:Td}
    {_6^{0}}T_d=\begin{bmatrix}
        r_{11} & r_{12} & r_{13} & q_x \\
        r_{21} & r_{22} & r_{23} & q_y \\
        r_{31} & r_{32} & r_{33} & q_z \\
        0      & 0      & 0      & 1
    \end{bmatrix}
\end{equation}
where $q_x$, $q_y$ and $q_z$ are the desired end-effector position. We will use Equation \ref{eq:inv}
to solve for the joint angles.
\begin{equation} \label{eq:inv}
    {_6^{0}}T={_6^{0}}T_d
\end{equation}
We can manipulate Equation \ref{eq:inv} to isolate each joint angle. To do that we need to find the inveses of some intermediate homogenous transformation matrices.
\begin{equation} \label{eq:TTinv}
    \begin{aligned}
        T      & =\begin{bmatrix}
                      n_x & o_x & a_x & q_x \\
                      n_y & o_y & a_y & q_y \\
                      n_z & o_z & a_z & q_z \\
                      0   & 0   & 0   & 1
                  \end{bmatrix}                   \\
        T^{-1} & =\begin{bmatrix}
                      n_x & n_y & n_z & -n_xq_x-n_yq_y-n_zq_z \\
                      o_x & o_y & o_z & -o_xq_x-o_yq_y-o_zq_z \\
                      a_x & a_y & a_z & -a_xq_x-a_yq_y-a_zq_z \\
                      0   & 0   & 0   & 1
                  \end{bmatrix}
    \end{aligned}
\end{equation}

% theta_1
\subsection{Solution for $\theta_1$}
Equation \ref{eq:T40}, \ref{eq:T10}, and \ref{eq:TTinv} are used to isolate $\theta_1$ as follows.
\begin{equation} \label{eq:t11}
    \begin{aligned}
        {_0^{1}}T{_4^{0}}T & ={_0^{1}}T{_6^{0}}T_d{_5^{6}}T{_4^{5}}T \\
        \implies{_4^{1}}T  & ={_4^{1}}T_d
    \end{aligned}
\end{equation}
The matrices are too big and are not written here. The sybolic analysis done using MATLAB is attached together with this report.
Equating ${_4^{1}}T_{2,4}={_4^{1}}T_{d2,4}$ results:
\begin{equation} \label{eq:t12}
    \begin{aligned}
         & q_yc_1 - q_xs_1 - d_6r_{23}c_1 + d_6r_{13}s_1=0               \\
         & c_1\left(q_y-d_6r_{23}\right) = s_1\left(q_x-d_6r_{13}\right) \\
         & \frac{s_1}{c_1}=\frac{q_y-d_6r_{23}}{q_x-d_6r_{13}}
    \end{aligned}
\end{equation}
Therefore,
\begin{equation} \label{eq:t13}
    \theta_1=\mathrm{atan2}\left(q_y-d_6r_{23},q_x-d_6r_{13}\right)
\end{equation}
To avoid sholder singularity, the following condition must be satiesfied.
\begin{equation} \label{eq:t1_singularity}
    q_y-d_6r_{23}\neq0 \land q_x-d_6r_{13}\neq0
\end{equation}

% theta_2
\subsection{Solution for $\theta_2$}
We will follow the same procedure as in the previous section.
Equation \ref{eq:t11} will be used to solve for $\theta_2$.
\begin{equation} \label{eq:t21}
    \begin{aligned}
        {_4^{1}}T_{1,4}                & ={_4^{1}}T_{d1,4}                              \\
        a_3c_{23} - d_4s_{23} + a_2c_2 & =q_xc_1 + q_ys_1 - d_6r_{13}c_1 - d_6r_{23}s_1
    \end{aligned}
\end{equation}
Let $E=q_xc_1 + q_ys_1 - d_6r_{13}c_1 - d_6r_{23}s_1$
\begin{equation} \label{eq:t22}
    a_3c_{23} - d_4s_{23} + a_2c_2=E
\end{equation}

\begin{equation} \label{eq:t23}
    \begin{aligned}
        {_4^{1}}T_{3,4}                 & ={_4^{1}}T_{d3,4}      \\
        -d_4c_{23} - a_3s_{23} - a_2s_2 & =q_z - d_1 - d_6r_{33}
    \end{aligned}
\end{equation}
Let $F=d_6r_{33}+d_1-q_z$
\begin{equation} \label{eq:t24}
    d_4c_{23} + a_3s_{23} + a_2s_2=F
\end{equation}
From Equation \ref{eq:t22} and \ref{eq:t24}, we can get the following new sets of equations.
\begin{equation} \label{eq:t25}
    \begin{aligned}
         & a_3c_{23} - d_4s_{23}=E-a_2c_2 \\
         & d_4c_{23} + a_3s_{23}=F-a_2s_2
    \end{aligned}
\end{equation}
Squaring both sides of Equation \ref{eq:t25} results:
\begin{equation}
    \begin{aligned}
         & a_3^2c_{23}^2 + d_4^2s_{23}^2-2a_3c_{23}d_4s_{23}=E^2+a_2^2c_2^2-2Ea_2c_2 \\
         & d_4^2c_{23}^2 + a_3^2s_{23}^2+2a_3c_{23}d_4s_{23}=F^2+a_2^2s_2^2-2Fa_2s_2
    \end{aligned}
\end{equation}
Adding Equation \ref{eq:t25} results:
\begin{equation} \label{eq:t26}
    \begin{aligned}
        a_3^2+d_4^2=E^2+F^2+a_2^2-2Ea_2c_2-2Fa_2s_2
    \end{aligned}
\end{equation}
Moving the unknowns to the left side of Equation \ref{eq:t26} results:
\begin{equation} \label{eq:t27}
    \begin{aligned}
        Ec_2+Fs_2=\frac{E^2+F^2+a_2^2-a_3^2-d_4^2}{2a_2}
    \end{aligned}
\end{equation}
Equation \ref{eq:t27} is obtained by assuming that $a_2\neq0$.
As can be seen from Equation \ref{eq:singular_config}, $a_2 = 0$ is one of
the singularity configurations. If $a_2=0$ then joint 2 and joint 3 will have the same
axis of rotation resulting in loss of one degree of freedom.\\
Let
\begin{equation} \label{eq:t28}
    K=\frac{E^2+F^2+a_2^2-a_3^2-d_4^2}{2a_2}
\end{equation}
Substituting Equation \ref{eq:t28} into Equation \ref{eq:t27} results:
\begin{equation} \label{eq:t29}
    Ec_2+Fs_2=K
\end{equation}
Equation \ref{eq:t29} can be solved by transforming it into polar form.
\begin{equation} \label{eq:t210}
    \begin{aligned}
         & r=\sqrt{E^2+F^2}                                            \\
         & \tan(\phi)=\frac{F}{E} \Rightarrow \phi=\mathrm{atan2}(F,E)
    \end{aligned}
\end{equation}
\begin{equation} \label{eq:t211}
    \begin{aligned}
        F & =rsin(\phi) \\
        E & =rcos(\phi)
    \end{aligned}
\end{equation}
Substituting Equation \ref{eq:t211} into Equation \ref{eq:t29} results:
\begin{equation} \label{eq:t212}
    \begin{aligned}
         & rcos(\phi)c_2+rsin(\phi)s_2=K \\
         & rcos(\phi-\theta_2)=K
    \end{aligned}
\end{equation}
Using Equation \ref{eq:t210} and Equation \ref{eq:t212} the value of $\theta_2$ is solved to be:
\begin{equation}
    \begin{aligned}
        \theta_2=\mathrm{atan2}(F,E) \pm \arccos\left(\frac{K}{sqrt{E^2+F^2}}\right)
    \end{aligned}
\end{equation}
Note, from Equation \ref{eq:singular_config} and Equation \ref{eq:t22} we can see that $E=0$ is
a singular confuguration and we need to take care of it on the programming.
% Theta 3
\subsection{Solution for $\theta_3$}
We can use Equation \ref{eq:T20}, Equation \ref{eq:T50}, Equation \ref{eq:Td}, and Equation \ref{eq:TTinv} are used to isolate $\theta_3$.
\begin{equation} \label{eq:t31}
    \begin{aligned}
        {_0^{2}}T{_5^{0}}T & ={_0^{2}}T{_6^{0}}T_d{_5^{6}}T \\
        {_5^{2}}T          & ={_5^{2}}T_d
    \end{aligned}
\end{equation}

\begin{equation} \label{eq:t32}
    \begin{aligned}
        {_5^{2}}T_{24}                                   & ={_5^{2}}T_{d24}                                        \\
        d_4c_3 + a_3s_3=d_1c_2 - q_zc_2 + d_6r_{33}c_2 - & q_xc_1s_2 - q_ys_1s_2 + d_6r_13c_1s_2 + d_6r_{23}s_1s_2
    \end{aligned}
\end{equation}
By rearranging Equation \ref{eq:t32} we can express it in terms of variable $E$ and $F$ that we have defined earier.
\begin{equation} \label{eq:t33}
    d_4c_3 + a_3s_3=c_2(d_6r_{33}+d_1-q_z)+s_2(c_1(d_6r_{13}-q_x) + s1(d_6r_{23}-q_y))
\end{equation}
Substituting $E$ and $F$ into Equation \ref{eq:t33} results:
\begin{equation} \label{eq:t34}
    d_4c_3 + a_3s_3=Fc_2-Es_2
\end{equation}
We can generate another equation using element $1,4$
\begin{equation} \label{eq:t35}
    \begin{aligned}
        {_5^{2}}T_{14}                                       & ={_5^{2}}T_{d14}                                            \\
        a_2 + a_3c_3 - d_4s_3=d_1s_2 - q_zs_2 + d_6r_{33}s_2 & + q_xc_1c_2 + q_yc_2s_1 - d_6r_{13}c_1c_2 - d_6r_{23}c_2s_1
    \end{aligned}
\end{equation}
By rearranging Equation \ref{eq:t35} we can express it in terms of variable $E$ and $F$ as well.
\begin{equation} \label{eq:t36}
    a_2 + a_3c_3 - d_4s_3=c_2(c_1(q_x-d_6r_{13}) + s_1(q_y-d_6r_{23})) + s_2(d_6r_{33}+d_1-q_z)
\end{equation}
Substituting $E$ and $F$ into Equation \ref{eq:t36} and moving $a_2$ to the right results:
\begin{equation} \label{eq:t37}
    a_3c_3 - d_4s_3=Ec_2+Fs_2-a_2
\end{equation}
Using the system of equation formed by Equation \ref{eq:t34} and Equation \ref{eq:t37} we can solve for $\theta_3$.
\begin{equation} \label{eq:t38}
    \begin{aligned}
        d_4c_3 + a_3s_3 & =Fc_2-Es_2     \\
        a_3c_3 - d_4s_3 & =Ec_2+Fs_2-a_2
    \end{aligned}
\end{equation}
Using variable change in $a_3$ and $d_4$ we can transform Equation \ref{eq:t38} into polar form and ulimately simplify
the system of equation.
\begin{equation} \label{eq:t39}
    \begin{aligned}
         & r=\sqrt{a_3^2+d_4^2}                                                 \\
         & \tan(\phi)=\frac{a_3}{d_4} \Rightarrow \phi=\mathrm{atan2}(a_3, d_4) \\
         & a_3=rsin(\phi)                                                       \\
         & d_4=rcos(\phi)
    \end{aligned}
\end{equation}
Substituting Equation \ref{eq:t39} into Equation \ref{eq:t38} results:
\begin{equation} \label{eq:t310}
    \begin{aligned}
        rc_\phi c_3 + rs_\phi s_3 & =Fc_2-Es_2     \\
        rs_\phi c_3 - rc_\phi s_3 & =Ec_2+Fs_2-a_2
    \end{aligned}
\end{equation}
Equation \ref{eq:t310} can be simplifies to:
\begin{equation} \label{eq:t311}
    \begin{aligned}
        rcos(\phi-\theta_3)   & =Fc_2-Es_2     \\
        rsin(\phi - \theta_3) & =Ec_2+Fs_2-a_2 \\
    \end{aligned}
\end{equation}
\begin{equation} \label{eq:t312}
    \tan\left(\phi - \theta_3\right)=\frac{Ec_2+Fs_2-a_2}{Fc_2-Es_2}
\end{equation}
Therefore,
\begin{equation} \label{eq:t313}
    \theta_3=\mathrm{atan2}(a_3, d_4) - \mathrm{atan2}(Ec_2+Fs_2-a_2, Fc_2-Es_2)
\end{equation}
One of the singularity confuguration term, $d_4c_3+a_3s_3$, appears while solving for $\theta_3$.
This fact is shown in Equation \ref{eq:t38}.
This singularity confuguration occures when the robot is fully stretched and it must as well be taken into
account when programming the robot.
% Theta 5
\subsection{Solution for $\theta_5$}
Equation \ref{eq:T30}, Equation \ref{eq:T50}, Equation \ref{eq:Td}, and Equation \ref{eq:TTinv} are used to isolate $\theta_5$.
\begin{equation} \label{eq:t51}
    \begin{aligned}
        {_0^{3}}T{_5^{0}}T & ={_0^{3}}T{_6^{0}}T_d{_5^{6}}T \\
        {_5^{3}}T          & ={_5^{3}}T_d
    \end{aligned}
\end{equation}

\begin{equation} \label{eq:t52}
    \begin{aligned}
        {_5^{3}}T_{22}                    & ={_5^{3}}T_{d22}                                                      \\
        c_5=r_{33}s_2s_3 - r_{33}c_2c_3 - & r_{13}c_1c_2s_3 - r_{13}c_1c_3s_2 - r_{23}c_2s_1s_3 - r_{23}c_3s_1s_2
    \end{aligned}
\end{equation}
Equation \ref{eq:t52} can be simplified to:
\begin{equation} \label{eq:t53}
    c_5=-s_{23}(r_{13}c_1+r_{23}s_1)-r_{33}c_{23}
\end{equation}
Let
\begin{equation} \label{eq:t54}
    A=r_{13}c_1+r_{23}s_1
\end{equation}
\begin{equation} \label{eq:t55}
    c_5=-r_{33}c_{23}-As_{23}
\end{equation}
Using the trigonometric identity $c_5^2+s_5^2=1$ we can solve for $s_5$ as follows:
\begin{equation} \label{eq:t56}
    s_5=\pm\sqrt{1-c_5^2}
\end{equation}
Therefore,
\begin{equation} \label{eq:t57}
    \begin{aligned}
        \theta_5 & =\mathrm{atan2}(s_5, c_5)                                                                           \\
        \theta_5 & =\pm\mathrm{atan2}\left(\sqrt{1-\left(-r_{33}c_{23}-As_{23}\right)^2}, -r_{33}c_{23}-As_{23}\right)
    \end{aligned}
\end{equation}

% Theta 4
\subsection{Solution for $\theta_4$}
Equation \ref{eq:t51} will be used to solve for $\theta_4$ as well.
\begin{equation} \label{eq:t41}
    \begin{aligned}
        {_5^{3}}T_{32} & ={_5^{3}}T_{d32}       \\
        s_4s_5         & =r_{23}c_1 - r_{13}s_1
    \end{aligned}
\end{equation}
Let $B=r_{23}c_1 - r_{13}s_1$
\begin{equation} \label{eq:t42}
    s_4s_5=B
\end{equation}
We can construct another Equation using element $1, 2$ of the same matrix as follows.
\begin{equation} \label{eq:t43}
    \begin{aligned}
        {_5^{3}}T_{12}                                        & ={_5^{3}}T_{d12}                                      \\
        -c_4s_5=r_{13}c_1c_2c_3 - r_{33}c_3s_2 - r_{33}c_2s_3 & + r_{23}c_2c_3s_1 - r_{13}c_1s_2s_3 - r_{23}s_1s_2s_3
    \end{aligned}
\end{equation}
Equation \ref{eq:t43} can be simplified to:
\begin{equation} \label{eq:t44}
    \begin{aligned}
        -c_4s_5=(r_{13}c_1+r_{23}s_1)c_{23}-r_{33}s_{23} \\
        \Rightarrow c_4s_5=r_{33}s_{23}-Ac_{23}
    \end{aligned}
\end{equation}
From Equation \ref{eq:t42} and Equation \ref{eq:t44} we can construct Equation \ref{eq:t45}.
\begin{equation} \label{eq:t45}
    \begin{aligned}
        s_4s_5 & =B                    \\
        c_4s_5 & =r_{33}s_{23}-Ac_{23}
    \end{aligned}
\end{equation}
Therefore, assuming that $s_5\neq0$, we can solve for $\theta_4$ as follows:
\begin{equation} \label{eq:t46}
    \theta_4=\mathrm{atan2}(B, r_{33}s_{23}-Ac_{23})
\end{equation}
Note that $s_5=0$ is one of the singularity configurations. We will solve for $\theta_4$ in this case later.
% Theta 6
\subsection{Solution for $\theta_6$}
We can isolate $\theta_6$ using Equation \ref{eq:T50}, Equation \ref{eq:T65}, Equation \ref{eq:Td}, and Equation \ref{eq:TTinv}.
\begin{equation} \label{eq:t61}
    \begin{aligned}
        {_6^{5}}T & ={_0^{5}}T{_6^{0}}T_d \\
        {_6^{5}}T & ={_6^{5}}T_d
    \end{aligned}
\end{equation}
\begin{equation} \label{eq:t62}
    \begin{split}
        {_6^{5}}T_{11}&={_6^{5}}T_{d11}\\
        c_6=r_{11}c_5s_1s_4 - r_{31}s_{23}c_4c_5 &- r_{11}s_{23}c_1s_5 - r_{21}s_{23}s_1s_5 - r_{21}c_1c_5s_4 \\
        &- r_{31}c_{23}s_5 + r_{21}c_{23}c_4c_5s_1 + r_{11}c_{23}c_1c_4c_5
    \end{split}
\end{equation}
Collecting like terms of Equation \ref{eq:t62} results:
\begin{equation} \label{eq:t63}
    \begin{aligned}
        c_6= & r_{11}\left(c_5s_1s_4+c_1(c_2(c_3c_4c_5-s_3s_5)-s_2(c_4c_5s_3+c_3s_5))\right)       \\
             & -r_{21}\left(c_4c_5s_1s_2s_3+c_1c_5s_4+c_3s_1s_2s_5+c_2s_1(s_3s_5-c_3c_4c_5)\right) \\
             & +r_{31}\left(s_3(s_2s_5-c_2c_4c_5)-c_3(c_4c_5s_2+c_2s_5)\right)
    \end{aligned}
\end{equation}
Using element $1, 2$ of the same matrix we can construct another equation as follows:
\begin{equation} \label{eq:t64}
    \begin{split}
        {_6^{5}}T_{12}&={_6^{5}}T_{d12}\\
        -s_6=r_{12}c_5s_1s_4 - r_{32}s_{23}c_4c_5 &- r_{12}s_{23}c_1s_5 - r_{22}s_{23}s_1s_5 - r_{22}c_1c_5s_4 \\
        &- r_{32}c_{23}s_5 + r_{22}c_{23}c_4c_5s_1 + r_{12}c_{23}c_1c_4c_5
    \end{split}
\end{equation}
Collecting like terms of Equation \ref{eq:t64} results:
\begin{equation} \label{eq:t65}
    \begin{aligned}
        -s_6=&r_{12}(c_5s_1s_4+c_1(c_2(c_3c_4c_5-s_3s_5)-s_2(c_4c_5s_3+c_3s_5)))\\
        &-r_{22}(c_4c_5s_1s_2s_3 + c_1c_5s_4 + c_3s_1s_2s_5 + c_2s_1 (s_3s_5 - c_3c_4c_5))\\
        &+r_32(s_3(s_2s_5-c_2c_4c_5)-c_3(c_4c_5s_2+c_2s_5))
    \end{aligned}
\end{equation}
Therefore, using Equation \ref{eq:t63} and Equation \ref{eq:t65} $\theta_6$ can be solved as follows:
\begin{equation} \label{eq:t66}
    \theta_6=\mathrm{atan2}(s_6,c_6)
\end{equation}
% Theta 6 and 4 when s5=0
Now let's solve for $\theta_4$ and $\theta_6$ when $s_5=0$, i.e., when there is a wrist singularity.
This condition results when $z_4$ and $z_6$ coincides. Resulting in cancellation of motion of joint 4 and joint 6.
Equation \ref{eq:T10}, Equation \ref{eq:T60}, Equation \ref{eq:Td}, and Equation \ref{eq:TTinv} are used to isolate $\theta_4$ and $\theta_6$.
\begin{equation} \label{eq:t641}
    \begin{aligned}
        {_0^{1}}T{_6^{0}}T & ={_0^{1}}T{_6^{0}}T_d \\
        {_6^{1}}T & ={_6^{1}}T_d
    \end{aligned}
\end{equation}
\begin{equation} \label{eq:t642}
    \begin{aligned}
        {_6^{1}}T_{21} & ={_6^{1}}T_{d21}\\
        -c_4s_6 - c_5c_6s_4 &= r_{21}c_1 - r_{11}s_1
    \end{aligned}
\end{equation}
Using element $2, 2$ of the same matrix we can construct another equation as follows:
\begin{equation} \label{eq:t643}
    \begin{aligned}
        {_6^{1}}T_{22} & ={_6^{1}}T_{d22}\\
        c_5s_4s_6 - c_4c_6 &= r_{22}c_1 - r_{12}s_1
    \end{aligned}
\end{equation}
Simplifying Equation \ref{eq:t642} and Equation \ref{eq:t643} results Equation \ref{eq:t644}.
\begin{equation} \label{eq:t644}
    \begin{aligned}
        -\sin(\theta_6\pm\theta_4) &= r_{21}c_1 - r_{11}s_1 \\
        -\cos(\theta_6\pm\theta_4) &= r_{22}c_1 - r_{12}s_1
    \end{aligned}
\end{equation}
In wrist singularity confuguration $\theta_4$ and $\theta_6$ produces a motion that will effectively cancel each other. 
Therefore, it is logical to fix the value of one of those angles to its previous value ans solve for the other.\\
By fixing $\theta_4$ to its previous value, $\theta_{4prev}$, we can solve for $\theta_6$ as follows:
\begin{equation} \label{eq:t645}
    \begin{aligned}
        \theta_4&=\theta_{4prev}\\
        \theta_6&=\mathrm{atan2}\left(-(r_{21}c_1 - r_{11}s_1), -(r_{22}c_1 - r_{12}s_1)\right)\mp\theta_{4prev}
    \end{aligned}
\end{equation}
